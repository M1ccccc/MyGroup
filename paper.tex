\documentclass{article}
\usepackage{fontspec}
\setmainfont{Times New Roman}

\title{Группа X}

\begin{document}
\section*{Владислав Мануковский}

Быстрое преобразование Фурье (БПФ, FFT — Fast Fourier Transform)
Это алгоритм для быстрого вычисления **дискретного преобразования Фурье (ДПФ).

1. Что такое ДПФ?  
Дискретное преобразование Фурье раскладывает сигнал (например, аудио, изображение, временной ряд) на сумму синусоид разных частот.  
Формула ДПФ для последовательности \( x_0, x_1, \dots, x_{N-1} \):  

\[
X_k = \sum_{n=0}^{N-1} x_n \cdot e^{-i \frac{2\pi}{N} kn}, \quad k = 0, 1, \dots, N-1
\]  

Где:  
- \( X_k \) — коэффициенты Фурье (амплитуды частот),  
- \( N \) — количество точек,  
- \( e^{-i \frac{2\pi}{N} kn} \) — комплексная экспонента (вращающийся фазовый множитель).  

2. Проблема: ДПФ медленное  
Прямое вычисление ДПФ требует \( O(N^2) \) операций (для каждой из \( N \) частот — сумма по \( N \) точкам).  

3. Как ускоряет БПФ? 
БПФ использует разделяй и властвуй и симметрию ДПФ, уменьшая сложность до \( O(N \log N) \).  

Основные идеи:  
- Разбиение на чётные и нечётные компоненты:  
\[
X_k = \sum_{m=0}^{N/2-1} x_{2m} \cdot e^{-i \frac{2\pi}{N} (2m)k} + \sum_{m=0}^{N/2-1} x_{2m+1} \cdot e^{-i \frac{2\pi}{N} (2m+1)k}
\]  
Это превращает одну задачу размера \( N \) в две задачи размера \( N/2 \).  

- Рекурсия: Процесс повторяется, пока не останутся маленькие ДПФ (например, из 2 точек).  

- Комбинирование результатов: Коэффициенты для полного ДПФ получаются из половинных результатов.  

4. Где применяется БПФ?  
- Обработка сигналов: Аудио, радиоволны, сжатие данных (MP3, JPEG).  
- Телекоммуникации: Модуляция/демодуляция (OFDM в Wi-Fi, 4G/5G).  
- Физика и инженерия: Анализ колебаний, спектроскопия.  
- Машинное обучение: Быстрые свёртки (например, в нейросетях).  

5. Пример работы БПФ 
Пусть есть сигнал из 4 точек: \( [x_0, x_1, x_2, x_3] \).  
- Разбиваем на чётные \( [x_0, x_2] \) и нечётные \( [x_1, x_3] \).  
- Вычисляем ДПФ для каждой половины.  
- Комбинируем результаты с учётом поворотных коэффициентов.  

Вывод  
БПФ — это оптимизированный способ перейти из временной области в частотную, который работает в сотни раз быстрее прямого расчёта ДПФ.  

\section*{Ярослав Лузан}

\section*{Михаил Веселов}

Соблюдаем установленные правила на моей ветке!
Новые правила: не пить, не курить

\section*{Сергей Ушаков}

Приветствую в своей ветке. Надеюсь вам тут понравится :D qq 12345

\section*{Никита Кутырев}
Я великий Кут
\section*{Владислав Кипаренко}
Привет
\section*{Ян Котенко}
Я хочу пельмени со свининой
\section*{Яна Крапивина}

поляниця
\section*{Лиза Хоменко}
спонсор пары - Кузиновский завод

\section*{Вова Хомяков}
Ибо Бог не дал нам духа робости и трусости, малодушия, угрюмости и заискивающего страха, но Он дал нам дух силы, любви, спокойствия и уравновешенности мыслей, самоконтроля и дисциплины.
\section*{Олеся Черняева}
по бокам конвой

\section*{Катя Морозова}
аттестат в крови

\section*{Лана Крапивина}

\section*{Ромиш Курбонов}

Мой дядя самых честных правил, Когда не в шутку занемог, Он уважать себя заставил И лучше выдумать не мог.
Его пример другим наука; Но, боже мой, какая скука С больным сидеть и день и ночь, Не отходя ни шагу прочь!
Какое низкое коварство Полуживого забавлять, Ему подушки поправлять, Печально подносить лекарство,
Вздыхать и думать про себя: Когда же черт возьмет тебя!

\section*{Никита Ганенко}

Всем привет, меня зовут Никита, и я хотел бы рассказать о трагичной истории нашей группы, это история ужасна и страшна, я считаю, что каждый должен её услышать и запомнить для себя её итоги и сделать выводы. Это история не так проста и мне почему-то кажется что даже самый матёр литеравед не смог бы в этом разобраться, его бы поглотило отчаяние и разочарование. Но всё же я думаю выходом из данной ситуации можно было бы попробовать применить один из методов, который бы с 95% случае нам помог, но всё же эти жалкие 5% могут всё разрушить и сделать шанс на хорогий исход бесмыслицей. Вообщем эта история о храбрости, подвигах и в тоже время об ужасающих временах наступивших в группе, это история об......
--
\section*{Никита Шиляев}

HOPOPOPOPOPOPOPOPOPOPOPO


\section *{Даниил Полянцев}
Вот это я крутой
\section*{Андрей Ромахин}

\section*{Михаил Шляхецкий}

Подтверждаю! Дублирую подтверждение!!!!!

\section*{Денис Медведев}

Чебурашку забрали в армию. Через два года он возвращается. Гена спрашивает:
— Кем служил, Чебурашка?
— Радаром.

© https://anekdoty.ru/
\section*{Александр Матвеев}

Всем привет!


\section*{Дмитрий Кочнев}
хакер на минималках
\section*{Данил Суворов}
Рассказываю что-то про себя.
\section*{Константин Малютин}
 15к ммр
\section*{Владислав Козионов}

\section*{Артем Васькив}
Описание о себе
\section*{Артем Консевич}
1к ммр
\section*{Владимир Рыбалка}

\end{document}
