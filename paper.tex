\documentclass{article}
\usepackage{fontspec}
\setmainfont{Times New Roman}

\title{Группа X}

\begin{document}
\section*{Владислав Мануковский}

Быстрое преобразование Фурье (БПФ, FFT — Fast Fourier Transform)
Это алгоритм для быстрого вычисления **дискретного преобразования Фурье (ДПФ).

1. Что такое ДПФ?  
Дискретное преобразование Фурье раскладывает сигнал (например, аудио, изображение, временной ряд) на сумму синусоид разных частот.  
Формула ДПФ для последовательности \( x_0, x_1, \dots, x_{N-1} \):  

\[
X_k = \sum_{n=0}^{N-1} x_n \cdot e^{-i \frac{2\pi}{N} kn}, \quad k = 0, 1, \dots, N-1
\]  

Где:  
- \( X_k \) — коэффициенты Фурье (амплитуды частот),  
- \( N \) — количество точек,  
- \( e^{-i \frac{2\pi}{N} kn} \) — комплексная экспонента (вращающийся фазовый множитель).  

2. Проблема: ДПФ медленное  
Прямое вычисление ДПФ требует \( O(N^2) \) операций (для каждой из \( N \) частот — сумма по \( N \) точкам).  

3. Как ускоряет БПФ? 
БПФ использует разделяй и властвуй и симметрию ДПФ, уменьшая сложность до \( O(N \log N) \).  

Основные идеи:  
- Разбиение на чётные и нечётные компоненты:  
\[
X_k = \sum_{m=0}^{N/2-1} x_{2m} \cdot e^{-i \frac{2\pi}{N} (2m)k} + \sum_{m=0}^{N/2-1} x_{2m+1} \cdot e^{-i \frac{2\pi}{N} (2m+1)k}
\]  
Это превращает одну задачу размера \( N \) в две задачи размера \( N/2 \).  

- Рекурсия: Процесс повторяется, пока не останутся маленькие ДПФ (например, из 2 точек).  

- Комбинирование результатов: Коэффициенты для полного ДПФ получаются из половинных результатов.  

4. Где применяется БПФ?  
- Обработка сигналов: Аудио, радиоволны, сжатие данных (MP3, JPEG).  
- Телекоммуникации: Модуляция/демодуляция (OFDM в Wi-Fi, 4G/5G).  
- Физика и инженерия: Анализ колебаний, спектроскопия.  
- Машинное обучение: Быстрые свёртки (например, в нейросетях).  

5. Пример работы БПФ 
Пусть есть сигнал из 4 точек: \( [x_0, x_1, x_2, x_3] \).  
- Разбиваем на чётные \( [x_0, x_2] \) и нечётные \( [x_1, x_3] \).  
- Вычисляем ДПФ для каждой половины.  
- Комбинируем результаты с учётом поворотных коэффициентов.  

Вывод  
БПФ — это оптимизированный способ перейти из временной области в частотную, который работает в сотни раз быстрее прямого расчёта ДПФ.  

\section*{Ярослав Лузан}

\section*{Михаил Веселов}

Соблюдаем установленные правила на моей ветке!
Новые правила: не пить, не курить

\section*{Сергей Ушаков}

Приветствую в своей ветке. Надеюсь вам тут понравится :D qq 12345

\section*{Никита Кутырев}
Я великий Кут
\section*{Владислав Кипаренко}
Привет
\section*{Ян Котенко}
Я хочу пельмени со свининой
\section*{Яна Крапивина}

\section*{Ромиш Курбонов}

\section*{Дмитрий Кочнев}

\end{document}
