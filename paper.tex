\documentclass{article}
\usepackage{fontspec}
\setmainfont{Times New Roman}

\title{Группа X}

\begin{document}
\section*{Владислав Мануковский}
Привет.
\section*{Ярослав Лузан}

\section*{Михаил Веселов}

\section*{Яна Крапивина}

\section*{Ромиш Курбонов}

Мой дядя самых честных правил, Когда не в шутку занемог, Он уважать себя заставил И лучше выдумать не мог.
Его пример другим наука; Но, боже мой, какая скука С больным сидеть и день и ночь, Не отходя ни шагу прочь!
Какое низкое коварство Полуживого забавлять, Ему подушки поправлять, Печально подносить лекарство,
Вздыхать и думать про себя: Когда же черт возьмет тебя!

\section*{Никита Ганенко}

Всем привет, меня зовут Никита, и я хотел бы рассказать о трагичной истории нашей группы, это история ужасна и страшна, я считаю, что каждый должен её услышать и запомнить для себя её итоги и сделать выводы. Это история не так проста и мне почему-то кажется что даже самый матёр литеравед не смог бы в этом разобраться, его бы поглотило отчаяние и разочарование. Но всё же я думаю выходом из данной ситуации можно было бы попробовать применить один из методов, который бы с 95% случае нам помог, но всё же эти жалкие 5% могут всё разрушить и сделать шанс на хорогий исход бесмыслицей. Вообщем эта история о храбрости, подвигах и в тоже время об ужасающих временах наступивших в группе, это история об......
--
\section*{Никита Шиляев}

HOPOPOPOPOPOPOPOPOPOPOPO

\section*{Андрей Ромахин}

\section*{Михаил Шляхецкий}

Подтверждаю! Дублирую подтверждение!!!!!

\section*{Денис Медведев}

Чебурашку забрали в армию. Через два года он возвращается. Гена спрашивает:
— Кем служил, Чебурашка?
— Радаром.

© https://anekdoty.ru/
\section*{Александр Матвеев}

\section*{Дмитрий Кочнев}

\end{document}
